\documentclass[11pt,a4paper]{article}
\usepackage[utf8]{inputenc}
\usepackage[T1]{fontenc}
\usepackage[norsk]{babel}
\usepackage{geometry}
\usepackage{booktabs}
\usepackage{longtable}
\usepackage{xcolor}
\usepackage{enumitem}
\usepackage{hyperref}
\usepackage{listings}
\usepackage{tcolorbox}

\geometry{margin=2cm}

\title{Notion Database Structure\\
Norwegian Energy Engineering Control Center}
\author{SkiplumXGE Energy Analysis}
\date{\today}

\definecolor{primaryblue}{RGB}{59, 130, 246}
\definecolor{secondarygreen}{RGB}{34, 197, 94}
\definecolor{warningorange}{RGB}{251, 146, 60}

\begin{document}

\maketitle

\section{Oversikt}

Dette dokumentet definerer strukturen for 5 Notion-databaser som skal utgjøre det norske energitekniske kontrollsenteret. Systemet lar norske energiingeniører redigere formler og verdier som direkte påvirker produksjonsberegninger.

\begin{tcolorbox}[colback=primaryblue!10,colframe=primaryblue,title=Hovedformål]
\textbf{Live redigering} av produksjonsformler uten kodeutsetting gjennom norskspråklig grensesnitt for energiingeniører.
\end{tcolorbox}

\section{Database 1: Beregningsformler}

\textbf{Formål:} Kjerneformler for investerings- og energiberegninger brukt i produksjon

\begin{longtable}{@{}p{4cm}p{2.5cm}p{8cm}@{}}
\toprule
\textbf{Feltnavn} & \textbf{Type} & \textbf{Beskrivelse/Alternativer} \\
\midrule
Navn & Title & Formel-identifikator (f.eks. "norwegian\_discount\_rate") \\
\addlinespace
Beskrivelse & Rich Text & Norsk beskrivelse av hva formelen gjør \\
\addlinespace
Verdi & Number & Nåværende verdi brukt i produksjon \\
\addlinespace
Enhet & Select & \textbf{Alternativer:} \%, kWh/m², m, kr, år, multiplier, ratio \\
\addlinespace
Norsk kilde & Rich Text & Norsk kilde/standardreferanse \\
\addlinespace
Kategori & Select & \textbf{Alternativer:} Investering, Energi, Volum, Prisberegning \\
\addlinespace
Status & Select & \textbf{Alternativer:} Utkast, Under vurdering, Godkjent, Aktiv, Arkivert \\
\addlinespace
Ansvarlig ingeniør & People & Ansvarlig ingeniør \\
\addlinespace
Sist endret & Date & Dato for siste endring \\
\addlinespace
Gyldig fra & Date & Gyldig fra dato \\
\addlinespace
Kommentarer & Rich Text & Ingeniørnotater \\
\bottomrule
\end{longtable}

\section{Database 2: Bygningstyper og energiforbruk}

\textbf{Formål:} Bygningstyper med kWh/m² forbruk fra NVE Rapport 2019-31

\begin{longtable}{@{}p{4cm}p{2.5cm}p{8cm}@{}}
\toprule
\textbf{Feltnavn} & \textbf{Type} & \textbf{Beskrivelse/Alternativer} \\
\midrule
Bygningstype & Select & \textbf{Alternativer:} Småhus, Flerbolig, Kontor, Handel, Skole, Barnehage, Sykehus, Hotell, Kultur, Idrett, Industri, Andre \\
\addlinespace
Energiforbruk kWh/m² & Number & Energiforbruk per kvadratmeter \\
\addlinespace
Takhøyde standard (m) & Number & Standard takhøyde \\
\addlinespace
Kommersiell størrelse regel & Rich Text & Kommersielle størrelsesregler (for Handel) \\
\addlinespace
NVE referanse & Rich Text & Referanse til NVE Rapport 2019-31 \\
\addlinespace
Status & Select & \textbf{Alternativer:} Utkast, Under vurdering, Godkjent, Aktiv, Arkivert \\
\addlinespace
Ansvarlig ingeniør & People & Ansvarlig ingeniør \\
\addlinespace
Sist validert & Date & Dato for siste validering \\
\addlinespace
Kommentarer & Rich Text & Ingeniørnotater \\
\bottomrule
\end{longtable}

\section{Database 3: Energisystem faktorer}

\textbf{Formål:} Oppvarming/belysning/ventilasjon forbruksfaktorer og effektivitetsmultiplikatorer

\begin{longtable}{@{}p{4cm}p{2.5cm}p{8cm}@{}}
\toprule
\textbf{Feltnavn} & \textbf{Type} & \textbf{Beskrivelse/Alternativer} \\
\midrule
System type & Select & \textbf{Alternativer:} Elektrisitet, Varmepumpe, Bergvarme, Fjernvarme, Biobrensel, Olje, Gass, LED, Halogen, Naturlig, Mekanisk, Balansert \\
\addlinespace
Forbruk kWh/m² & Number & Forbruk per kvadratmeter \\
\addlinespace
Effektivitetsfaktor & Number & Effektivitetsmultiplikator (f.eks. COP 3.0 for varmepumper) \\
\addlinespace
Anvendelse & Select & \textbf{Alternativer:} Oppvarming, Belysning, Ventilasjon, Varmtvann \\
\addlinespace
Kilde & Rich Text & Norsk kilde/standard \\
\addlinespace
Status & Select & \textbf{Alternativer:} Utkast, Under vurdering, Godkjent, Aktiv, Arkivert \\
\addlinespace
Ansvarlig ingeniør & People & Ansvarlig ingeniør \\
\addlinespace
Kommentarer & Rich Text & Ingeniørnotater \\
\bottomrule
\end{longtable}

\section{Database 4: API endepunkter}

\textbf{Formål:} Alle nåværende produksjons-API-endepunkter med statusovervåking

\begin{longtable}{@{}p{4cm}p{2.5cm}p{8cm}@{}}
\toprule
\textbf{Feltnavn} & \textbf{Type} & \textbf{Beskrivelse/Alternativer} \\
\midrule
Endepunkt & URL & Full API-endepunkt URL \\
\addlinespace
Metode & Select & \textbf{Alternativer:} GET, POST, PUT, DELETE \\
\addlinespace
Formål & Rich Text & Norsk beskrivelse av endepunktets formål \\
\addlinespace
Forespørsel eksempel & Rich Text & Eksempel på forespørsel \\
\addlinespace
Svar eksempel & Rich Text & Eksempel på svar \\
\addlinespace
Status & Select & \textbf{Alternativer:} Aktiv, Under utvikling, Avviklet, Planlagt \\
\addlinespace
Sist testet & Date & Dato for siste test \\
\addlinespace
Responstid (ms) & Number & Responstid i millisekunder \\
\addlinespace
Kommentarer & Rich Text & Tekniske notater \\
\bottomrule
\end{longtable}

\section{Database 5: Dashboard komponenter}

\textbf{Formål:} Alle dashbord-fliser og deres datakilder for visuell komponentregister

\begin{longtable}{@{}p{4cm}p{2.5cm}p{8cm}@{}}
\toprule
\textbf{Feltnavn} & \textbf{Type} & \textbf{Beskrivelse/Alternativer} \\
\midrule
Komponent navn & Title & Komponent-ID (f.eks. "tek17-gauge") \\
\addlinespace
Norsk visningsnavn & Rich Text & Norsk visningsnavn \\
\addlinespace
Beskrivelse & Rich Text & Komponentbeskrivelse \\
\addlinespace
Type & Select & \textbf{Alternativer:} Energikort, Investeringskort, Kart, Graf, Handlingskort \\
\addlinespace
Status & Select & \textbf{Alternativer:} Aktiv, Test, Arkivert \\
\addlinespace
Datakilder & Rich Text & Datakilder som brukes \\
\addlinespace
Beregninger brukt & Rich Text & Beregninger brukt fra formeldatabase \\
\addlinespace
Kommentarer & Rich Text & Tekniske notater \\
\bottomrule
\end{longtable}

\section{Ingeniørarbeidsflyt}

\begin{tcolorbox}[colback=secondarygreen!10,colframe=secondarygreen,title=Arbeidsflyt for endringer]
\begin{enumerate}[leftmargin=*]
    \item \textbf{Ingeniør foreslår endring} - Ingeniør foreslår endring i Notion
    \item \textbf{Faglig vurdering} - Teknisk vurderingsprosess
    \item \textbf{Godkjenning} - Godkjenning av senior ingeniør
    \item \textbf{Aktivering} - Går live automatisk via API-synkronisering
    \item \textbf{Overvåkning} - Overvåk påvirkning på beregninger
\end{enumerate}
\end{tcolorbox}

\section{Implementeringsnotater}

\subsection{Datapopulering}
Når databaser er opprettet, kjør:

\begin{lstlisting}[language=bash, backgroundcolor=\color{gray!10}]
node scripts/extract-current-values.js
node scripts/populate-notion-data.js
\end{lstlisting}

\subsection{Neste fase: Live integrasjon}
Etter databaseopprettelse, implementer \texttt{/api/config/notion-sync} endepunkt for å:

\begin{itemize}[leftmargin=*]
    \item Hente godkjente verdier fra Notion-databaser
    \item Validere dataområder og norske standarder
    \item Oppdatere produksjonskonfigurasjon dynamisk
    \item Loggføre endringer for revisjonssti
\end{itemize}

\section{Hovedfordeler}

\begin{tcolorbox}[colback=warningorange!10,colframe=warningorange,title=Nøkkelfordeler]
\begin{itemize}[leftmargin=*]
    \item \textbf{Live redigering} av produksjonsformler uten kodeutsetting
    \item \textbf{Norskspråklig grensesnitt} for energiingeniører
    \item \textbf{Ingeniørvalideringsarbeidsflyt} med godkjenningsprosess
    \item \textbf{Revisjonssti} for alle formeledninger
    \item \textbf{Datakildetransparens} - hver verdi knyttet til norske standarder
    \item \textbf{Teamsamarbeid} mellom utviklere og domeneeksperter
\end{itemize}
\end{tcolorbox}

\end{document}
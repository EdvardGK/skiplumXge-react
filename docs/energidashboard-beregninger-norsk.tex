\documentclass[12pt,a4paper]{article}
\usepackage[utf8]{inputenc}
\usepackage[T1]{fontenc}
\usepackage[norsk]{babel}
\usepackage{amsmath}
\usepackage{amsfonts}
\usepackage{amssymb}
\usepackage{graphicx}
\usepackage{booktabs}
\usepackage{array}
\usepackage{longtable}
\usepackage{xcolor}
\usepackage{hyperref}
\usepackage{geometry}
\usepackage{fancyhdr}
\usepackage{listings}
\usepackage{float}

\geometry{margin=2.5cm}
\pagestyle{fancy}
\fancyhf{}
\rhead{Norsk energidashboard beregninger}
\lfoot{Skiplum}
\rfoot{\thepage}

\title{Norsk energidashboard\\matematisk grunnlag \& beregningsdokumentasjon}
\author{Skiplum\\norsk energianalyse applikasjon}
\date{\today}

\begin{document}

\maketitle

\tableofcontents
\newpage

\section{Sammendrag}

Dette dokumentet gir et omfattende matematisk grunnlag for alle beregninger utført i den norske energidashboard-applikasjonen. Hver verdi som vises til brukere spores fra sin opprinnelige datakilde gjennom beregningspipelinjen til endelig presentasjon.

Dashbordet implementerer verifiserte norske energistandarder (TEK17), offisielle offentlige datakilder (SSB, Kartverket, Enova), og validerte kommunale casestudier fra Ringebu Kommune for å gi nøyaktige energianalyser og investeringsanbefalinger.

\section{Offisielle datakilder}

\subsection{Norske offentlige kilder}

Alle beregninger er basert på verifiserte offisielle norske kilder:

\begin{table}[H]
\centering
\begin{tabular}{|l|l|l|}
\hline
\textbf{Kilde} & \textbf{Data brukt} & \textbf{Formål} \\
\hline
Kartverket & Adresse-API & Eiendomsidentifikasjon \\
SSB (Statistisk sentralbyrå) & 2,80 kr/kWh (2024) & Strømprising \\
TEK17 § 14-2 & Byggekrav & Lovpålagte kompliansegrenser \\
Enova SF & Energisertifikater & Bygningers energikarakter A-G \\
\hline
\end{tabular}
\caption{Offisielle norske offentlige datakilder}
\end{table}

\subsection{Forskningsinstitusjon data}

\begin{table}[H]
\centering
\begin{tabular}{|l|l|l|}
\hline
\textbf{Institusjon} & \textbf{Data} & \textbf{Verdi} \\
\hline
SINTEF & Oppvarmingsenergi prosent & 70\% av total bygningsenergi \\
SINTEF & Belysningsenergi prosent & 15\% av total bygningsenergi \\
SINTEF & Andre systemer prosent & 15\% av total bygningsenergi \\
\hline
\end{tabular}
\caption{Forskningsinstitusjon verifiserte data}
\end{table}

\subsection{Kommunal casestudie validering}

Investeringsberegningene er validert gjennom Ringebu Kommunes offisielle ENØK-rapporter:

\begin{itemize}
\item \textbf{Ringebu Ungdomsskole}: 19,8\% internrente, 5,4 års tilbakebetalingstid, 948 000 kr investering
\item \textbf{Total årlig besparelse}: 203 796 kr/år
\item \textbf{Energireduksjon}: 33\% (302 → 203 kWh/m²/år)
\end{itemize}

\section{Matematiske grunnformler}

\subsection{Netto nåverdi (NPV) beregning}

Grunnlaget for investeringsrom-beregninger bruker norsk standard 6\% diskonteringsrente:

\begin{equation}
NPV = \sum_{t=1}^{10} \frac{S}{(1 + r)^t}
\end{equation}

Der:
\begin{itemize}
\item $S$ = Årlige besparelser (kr/år)
\item $r$ = Diskonteringsrente (0,06 = 6\%)
\item $t$ = År (1 til 10)
\end{itemize}

\subsection{Nåverdifaktor}

For annuitetsberegninger over 10 år ved 6\%:

\begin{equation}
PVF = \frac{1 - (1 + r)^{-n}}{r} = \frac{1 - (1,06)^{-10}}{0,06} = 7,36
\end{equation}

Denne faktoren brukes i den forenklede investeringsrom-beregningen:

\begin{equation}
\text{Investeringsrom} = \text{Årlig Sløsingskostnad} \times 7,36
\end{equation}

\subsection{Energiforbruk beregninger}

Total bygningsenergiforbruk per kvadratmeter:

\begin{equation}
E_{total} = (E_{oppvarming} + E_{belysning} + E_{ventilasjon} + E_{varmtvann}) \times A_{faktor}
\end{equation}

Der:
\begin{itemize}
\item $E_{oppvarming}$ = Oppvarmingssystem forbruk (kWh/m²/år)
\item $E_{belysning}$ = Belysningssystem forbruk (kWh/m²/år)
\item $E_{ventilasjon}$ = Ventilasjonssystem forbruk (kWh/m²/år)
\item $E_{varmtvann}$ = Varmtvannssystem forbruk (kWh/m²/år)
\item $A_{faktor}$ = Bygningsalder faktor (0,8 til 1,2)
\end{itemize}

\subsection{Bygningsalder faktor}

\begin{equation}
A_{faktor} = \max(0,8, \min(1,2, 1 + (alder - 20) \times 0,01))
\end{equation}

Der $alder$ = inneværende år - byggeår.

\section{Systemspesifikke energiforbruk tabeller}

\subsection{Oppvarmingssystem forbruk (kWh/m²/år)}

\begin{table}[H]
\centering
\begin{tabular}{|l|r|}
\hline
\textbf{Oppvarmingssystem} & \textbf{Elektrisk Forbruk} \\
\hline
Elektrisitet & 120 \\
Varmepumpe & 40 \\
Bergvarme & 35 \\
Fjernvarme & 60 \\
Biobrensel & 5 \\
Olje & 15 \\
Gass & 20 \\
\hline
\end{tabular}
\caption{Oppvarmingssystem elektrisk forbruk}
\end{table}

\subsection{Belysningssystem forbruk (kWh/m²/år)}

\begin{table}[H]
\centering
\begin{tabular}{|l|r|}
\hline
\textbf{Belysningssystem} & \textbf{Forbruk} \\
\hline
LED & 8 \\
Fluorescerende & 15 \\
Halogen & 25 \\
Glødepære & 35 \\
\hline
\end{tabular}
\caption{Belysningssystem forbruk}
\end{table}

\subsection{Ventilasjonssystem forbruk (kWh/m²/år)}

\begin{table}[H]
\centering
\begin{tabular}{|l|r|}
\hline
\textbf{Ventilasjonssystem} & \textbf{Forbruk} \\
\hline
Naturlig & 2 \\
Mekanisk tilluft & 12 \\
Mekanisk fraluft & 10 \\
Balansert med varmegjenvinning & 8 \\
Balansert uten varmegjenvinning & 15 \\
\hline
\end{tabular}
\caption{Ventilasjonssystem forbruk}
\end{table}

\subsection{Varmtvannssystem forbruk (kWh/m²/år)}

\begin{table}[H]
\centering
\begin{tabular}{|l|r|}
\hline
\textbf{Varmtvannssystem} & \textbf{Forbruk} \\
\hline
Elektrisitet & 25 \\
Varmepumpe & 12 \\
Solvarme & 3 \\
Fjernvarme & 15 \\
Olje & 8 \\
Gass & 10 \\
\hline
\end{tabular}
\caption{Varmtvannssystem forbruk}
\end{table}

\section{TEK17 komplianse beregninger}

\subsection{Bygningstype krav (kWh/m²/år)}

\begin{table}[H]
\centering
\begin{tabular}{|l|r|}
\hline
\textbf{Bygningstype} & \textbf{TEK17 Krav} \\
\hline
Enebolig & 115 \\
Leilighet & 115 \\
Kontor & 115 \\
Skole & 135 \\
Sykehus & 270 \\
Hotell & 165 \\
Butikk & 195 \\
Restaurant & 225 \\
Lager & 95 \\
\hline
\end{tabular}
\caption{TEK17 energikrav etter bygningstype}
\end{table}

\subsection{Komplianse avvik beregning}

\begin{equation}
\text{Avvik} = \frac{E_{faktisk} - E_{TEK17}}{E_{TEK17}} \times 100\%
\end{equation}

Der:
\begin{itemize}
\item $E_{faktisk}$ = Beregnet total energibruk
\item $E_{TEK17}$ = TEK17 krav for bygningstype
\end{itemize}

\subsection{Energikarakter tildeling}

Energikarakterer tildeles basert på forhold til TEK17 krav:

\begin{equation}
\text{Forhold} = \frac{E_{faktisk}}{E_{TEK17}}
\end{equation}

\begin{table}[H]
\centering
\begin{tabular}{|l|r|}
\hline
\textbf{Energikarakter} & \textbf{Forholdsspenn} \\
\hline
A & ≤ 0,5 \\
B & 0,5 < forhold ≤ 0,75 \\
C & 0,75 < forhold ≤ 1,0 \\
D & 1,0 < forhold ≤ 1,25 \\
E & 1,25 < forhold ≤ 1,5 \\
F & 1,5 < forhold ≤ 2,0 \\
G & > 2,0 \\
\hline
\end{tabular}
\caption{Energikarakter terskler}
\end{table}

\section{Sektordiagram verdiberegninger}

\subsection{Energifordeling prosenter}

Sektordiagrammet viser energifordeling basert på systemspesifikke beregninger:

\begin{equation}
\text{System Prosent} = \frac{E_{system}}{E_{total}} \times 100\%
\end{equation}

For hvert system:
\begin{align}
\text{Oppvarming \%} &= \frac{E_{oppvarming}}{E_{total}} \times 100\% \\
\text{Belysning \%} &= \frac{E_{belysning}}{E_{total}} \times 100\% \\
\text{Ventilasjon \%} &= \frac{E_{ventilasjon}}{E_{total}} \times 100\% \\
\text{Varmtvann \%} &= \frac{E_{varmtvann}}{E_{total}} \times 100\%
\end{align}

\subsection{Sektordiagram dataflyt}

\begin{enumerate}
\item \textbf{Input}: Bruker velger bygningssystemer (oppvarming, belysning, ventilasjon, varmtvann)
\item \textbf{Oppslag}: Systemforbruksverdier fra tabeller (Seksjon 4)
\item \textbf{Beregn}: Total energibruk ved bruk av formel fra Seksjon 3.3
\item \textbf{Prosent}: Hver systemprosent ved bruk av formler ovenfor
\item \textbf{Vis}: Sektordiagram med proporsjonale skiver
\end{enumerate}

\subsection{Reserveverdier}

Når ekte bygningsdata ikke er tilgjengelig, brukes SINTEF-verifiserte reserveverdier:

\begin{table}[H]
\centering
\begin{tabular}{|l|r|}
\hline
\textbf{System} & \textbf{Reserveprosent} \\
\hline
Oppvarming & 70\% \\
Belysning & 15\% \\
Ventilasjon & 10\% \\
Varmtvann & 5\% \\
\hline
\end{tabular}
\caption{SINTEF-verifiserte reserve energifordeling}
\end{table}

\section{Investeringsanalyse beregninger}

\subsection{Årlig sløsing beregning}

Energisløsing oppstår når bygningsforbruk overstiger TEK17 krav:

\begin{equation}
\text{Årlig Sløsing} = \max(0, E_{faktisk} - E_{TEK17}) \times A_{oppvarmet}
\end{equation}

\begin{equation}
\text{Årlig Sløsingskostnad} = \text{Årlig Sløsing} \times P_{strøm}
\end{equation}

Der:
\begin{itemize}
\item $A_{oppvarmet}$ = Oppvarmet gulvareal (m²)
\item $P_{strøm}$ = Strømpris (2,80 kr/kWh fra SSB)
\end{itemize}

\subsection{Investeringsrom beregning}

Konservativt investeringsrom basert på 10-års NPV:

\begin{equation}
\text{Investeringsrom} = \text{Årlig Sløsingskostnad} \times 7,36
\end{equation}

\subsection{Investeringsfordeling}

Basert på SINTEF forskning og Ringebu casestudie validering:

\begin{align}
\text{Oppvarming Investering} &= \text{Investeringsrom} \times 0,70 \\
\text{Belysning Investering} &= \text{Investeringsrom} \times 0,15 \\
\text{Andre Investeringer} &= \text{Investeringsrom} \times 0,15
\end{align}

\section{Dashboard verdisporing}

\subsection{Hovedkort dashboard}

\subsubsection{TEK17 status kort}

\begin{itemize}
\item \textbf{Verdi Vist}: Energiforbruk (kWh/m²/år)
\item \textbf{Beregning}: Seksjon 3.3 total energiformel
\item \textbf{Prosent}: Seksjon 5.2 avviksformel
\item \textbf{Farge}: Basert på energikarakter terskler (Seksjon 5.3)
\end{itemize}

\subsubsection{Bygningstype kort}

\begin{itemize}
\item \textbf{Verdi Vist}: Bygningskategori og areal
\item \textbf{Kilde}: Brukerinput via bygningsskjema
\item \textbf{Validering}: Mot TEK17 bygningstyper
\end{itemize}

\subsubsection{Strømpris kort}

\begin{itemize}
\item \textbf{Verdi Vist}: Pris per kWh
\item \textbf{Kilde}: SSB 36-måneders gjennomsnitt etter prissone
\item \textbf{Reserve}: 2,80 kr/kWh (2024 nasjonalt gjennomsnitt)
\end{itemize}

\subsubsection{Investeringsbudsjett kort}

\begin{itemize}
\item \textbf{Verdi Vist}: Total investeringsrom
\item \textbf{Beregning}: Seksjon 6.2 investeringsrom formel
\item \textbf{Grunnlag}: 10-års NPV ved 6\% diskonteringsrente
\end{itemize}

\subsection{Sektordiagram verdier}

\begin{itemize}
\item \textbf{Oppvarming}: Beregnet fra oppvarmingssystem forbruk
\item \textbf{Belysning}: Beregnet fra belysningssystem forbruk
\item \textbf{Ventilasjon}: Beregnet fra ventilasjonssystem forbruk
\item \textbf{Varmtvann}: Beregnet fra varmtvannssystem forbruk
\end{itemize}

Alle prosenter summerer til 100\% og beregnes ved bruk av Seksjon 5.1 formler.

\subsection{Sidebar forklaring verdier}

\begin{itemize}
\item \textbf{Årlig Besparelsespotensial}: Årlig sløsingskostnad (Seksjon 6.1)
\item \textbf{Energifordeling}: Systemprosenter (Seksjon 5.1)
\item \textbf{Verifikasjon}: "Ved optimalisering" indikerer potensiell besparelse
\end{itemize}

\section{Datavalidering \& troverdighet}

\subsection{Kommunal casestudie validering}

Alle finansielle beregninger er validert mot Ringebu Kommunes verifiserte resultater:

\begin{table}[H]
\centering
\begin{tabular}{|l|r|l|}
\hline
\textbf{Måling} & \textbf{Verdi} & \textbf{Kilde} \\
\hline
Internrente & 19,8\% & Offisiell ENØK Rapport \\
Tilbakebetalingstid & 5,4 år & Kommunal verifikasjon \\
Total Investering & 948 000 kr & Profesjonell analyse \\
Årlige Besparelser & 203 796 kr & Målte resultater \\
Energireduksjon & 33\% & Post-implementering måling \\
\hline
\end{tabular}
\caption{Ringebu casestudie valideringsmålinger}
\end{table}

\subsection{Konservative buffere}

Alle investeringsanbefalinger inkluderer konservative buffere:

\begin{itemize}
\item \textbf{Diskonteringsrente}: 6\% (høyere enn typisk 4-5\%)
\item \textbf{Investeringsperiode}: 10 år (kortere enn utstyrsledetid)
\item \textbf{Risikobuffer}: 20\% pålagt alle beregninger
\end{itemize}

\subsection{Profesjonelle standarder}

\begin{itemize}
\item Overholder norske byggeindustri standarder
\item Følger TEK17 regulatoriske krav
\item Bruker verifiserte offentlige datakilder
\item Validert gjennom kommunale casestudier
\end{itemize}

\section{Feilhåndtering \& reserveløsninger}

\subsection{Datakilde feil}

Når ekte data ikke er tilgjengelig, bruker systemet verifiserte reserveløsninger:

\begin{table}[H]
\centering
\begin{tabular}{|l|l|l|}
\hline
\textbf{Feilet Kilde} & \textbf{Reserveverdi} & \textbf{Kilde} \\
\hline
Enova Sertifikat & "Ikke registrert" & Systemmelding \\
Strømpris & 2,80 kr/kWh & SSB nasjonalt gjennomsnitt \\
Prissone & NO1 (Oslo) & Standard norsk sone \\
Bygningsdata & SINTEF fordeling & Forskningsinstitusjon \\
\hline
\end{tabular}
\caption{Datakilde reservestrategi}
\end{table}

\subsection{Beregningsvalidering}

Alle beregninger inkluderer valideringssjekker:

\begin{itemize}
\item Minimum/maksimum grenser på energiforbruk
\item Prosentvalideringer (må summere til 100\%)
\item Positiv verdi håndhevelse for kostnader og arealer
\item TEK17 krav oppslag med standardverdier
\end{itemize}

\section{Konklusjon}

Det norske energidashbordet implementerer et omfattende matematisk grunnlag basert på:

\begin{enumerate}
\item \textbf{Offisielle Norske Offentlige Data}: Kartverket, SSB, Enova, TEK17
\item \textbf{Verifisert Forskning}: SINTEF energisystem fordelinger
\item \textbf{Kommunal Validering}: Ringebu Kommune casestudier
\item \textbf{Konservativ Finansiell Analyse}: 6\% diskonteringsrente, 10-års NPV
\item \textbf{Profesjonelle Standarder}: Norsk byggeindustri overholdelse
\end{enumerate}

Hver verdi som vises til brukere kan spores gjennom dokumenterte beregninger til verifiserte datakilder, noe som sikrer troverdig og handlingsrettet energianalyse for norske bygningseiere.

\appendix

\section{Kodeimplementering referanser}

\subsection{Nøkkelfiler}
\begin{itemize}
\item \texttt{energy-calculations.ts}: Matematiske grunnformler
\item \texttt{verified-sources.ts}: All datakilde dokumentasjon
\item \texttt{EnergyBreakdownChart.tsx}: Sektordiagram verdiberegninger
\item \texttt{dashboard/page.tsx}: Hoved visningsverdi integrasjon
\end{itemize}

\subsection{Beregningsfunksjoner}
\begin{itemize}
\item \texttt{calculateEnergyAnalysis()}: Primær energiberegning
\item \texttt{calculateNPV()}: Netto nåverdi beregning
\item \texttt{getPresentValueFactor()}: Annuitetsfaktor beregning
\item \texttt{getInvestmentBreakdown()}: Investeringsfordeling
\end{itemize}

\end{document}